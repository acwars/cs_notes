\section{Present continuous (I am doing)}
\label{Present continuous}
$\textbf{am/is/are + -ing}$ is the present continuous.
\begin{flushleft}
\newcommand{\tabincell}[2]{\begin{tabular}{@{}#1@{}}#2\end{tabular}}
\begin{tabular}{ | l l | l | }
 \hline
 \tabincell{r}{I\\ he/she/it\\ we/you/they} & \tabincell{l}{\textbf{am}\\ \textbf{is}\\ \textbf{are}} & \tabincell{l}{\textbf{driving}\\ \textbf{working}\\ \textbf{doing} etc.} \\
 \hline
\end{tabular}
\end{flushleft}
I am doing something = I started doing it and I haven't finished.\\
I'm in the middle of doing it.\\
Somethimes the action is not happening at the time of speaking.\\
We use the continuous for things happening at or around the time of speaking.\\
The action is not complete.\\
We use the continuous for $\textit{temporary}$situations (things that continue for a short time). \\
\begin{tikzpicture}[]
    % draw horizontal line
    \draw[] (0,0) -- (10,0);

    % draw vertical lines
    \foreach \x in {5}
    \draw (\x cm,3pt) -- (\x cm,-3pt);

    % draw nodes
    \draw (0,0) node[below=3pt] {$ past $} node[above=3pt] {$   $};
    \draw (5,0) node[below=3pt] {$ now $} node[above=3pt] {$ \textbf{I am doing} $};
    \draw (10,0) node[below=3pt] {$ future $} node[above=3pt] {$ $};
\end{tikzpicture}

\section{Present simple (I do)}
\label{Present imple}
$\textbf{drive(s), work(s), do(es)}$ etc. is the \textit{present simple}
\begin{flushleft}
\begin{tabular}{ | l l | }
 \hline
 I/we/you/they & $\textbf{drive/work/do}$ etc. \\
 \hline
 he/she/it & $\textbf{drives/works/does}$ etc. \\
 \hline
\end{tabular}
\end{flushleft}
We use the present simple to talk about things in general.\\
We use it to say that something happens all the time or repeatedly, or that something is true in general.\\
We use $\textbf{do/does}$ to make questions sentences
\begin{flushleft}
\newcommand{\tabincell}[2]{\begin{tabular}{@{}#1@{}}#2\end{tabular}}
\begin{tabular}{ | l | l | l | }
 \hline
 \tabincell{c}{\textbf{do}\\\textbf{does}} & \tabincell{l}{I/we/you/they\\he/she/it} & \tabincell{l}{\textbf{work}?\\\textbf{drive}?\\\textbf{do}?} \\ 
 \hline
\end{tabular}
\end{flushleft}
We use $\textbf{do/does}$ to make negative sentences
\begin{flushleft}
\newcommand{\tabincell}[2]{\begin{tabular}{@{}#1@{}}#2\end{tabular}}
\begin{tabular}{ | l | l | l | }
 \hline
 \tabincell{c}{I/we/you/they\\ he/she/it} & \tabincell{l}{\textbf{don't}\\ \textbf{doesn't}} & \tabincell{l}{\textbf{work}\\ \textbf{drive}\\ \textbf{do}} \\
 \hline
\end{tabular}
\end{flushleft}
In the following examples, $\textbf{do}$ is also the main verb (do you $\textbf{do}$ / doesn't $\textbf{do}$ etc.) \\
We use the present simple to say how often we do things. \\
We use the simple for things in general or things that happen repeatedly. \\
We use the simple for $\textit{permanent}$situations (things that continue for a long time). \\
\begin{tikzpicture}[]
    % draw horizontal line
    \draw[] (0,0) -- (10,0);

    % draw nodes
    \draw (0,0) node[below=3pt] {$ past $} node[above=3pt] {$   $};
    \draw (5,0) node[below=3pt] {$ now $} node[above=3pt] {$ $};
    \draw (10,0) node[below=3pt] {$ future $} node[above=3pt] {$ $};
    \draw [black, ultra thick ,decorate,decoration={brace,amplitude=5pt}, yshift=-4pt] (0,0.5)  -- (10,0.5)  node [black,midway,above=4pt,xshift=-2pt] {\footnotesize $\textbf{I do}$};
\end{tikzpicture}

\section{Present perfect 1 (I have done)}
\label{Present perfect 1}
$\textbf{have lost / has lost}$ is the $\textit{present perfect simple}$ \\
The present perfect simple is $\textbf{have/has}$ + $\textbf{past participle}$. \\
\begin{flushleft}
\newcommand{\tabincell}[2]{\begin{tabular}{@{}#1@{}}#2\end{tabular}}
\begin{tabular}{ | l | l | l | }
 \hline
 \tabincell{c}{I/we/they/you\\ he/she/it} & \tabincell{l}{\textbf{have}\\ \textbf{has}} & \tabincell{l}{\textbf{finished}\\ \textbf{lost}\\ \textbf{done}\\ \textbf{been} etc.} \\
 \hline
\end{tabular}
\end{flushleft}
When we say `something $\textbf{has happened}$', this is usually new information.
When we use the present perfect, there is a connection with $\textit{now}$. The action in the past has a result $\textit{now}$
Compare $\textbf{gone (to)}$ and $\textbf{been (to)}$
\begin{itemize}
    \item[$\square$] He $\textbf{has gone to}$ (= he is there now or on his way there)
    \item[$\square$] She $\textbf{has been}$ (= she has now come back)
\end{itemize}
$\textbf{Just}$ = a short time ago \\
$\textbf{Already}$ = sooner than expected \\
$\textbf{Yet}$ = until now. We use $\textbf{yet}$ to show that we are expecting something to happen. \\

\section{Present perfect 2 (I have done)}
\label{Present perfect 2}
When we talk about a period of time that continues from the past until now, we use the $\textit{present perfect}$ ($\textbf{have been / have travelled}$ etc.).
$\textbf{been (to)}$ = visited
In the following examples too, the speakers are talking about a period that continues until now ($\textbf{recently, in the last few days, so far, since I arrived}$ etc.) \\
\begin{tikzpicture}[]
    % draw horizontal line
    \draw[] (0,0) -- (5,0);

    % draw nodes
    \draw (0,0) node[below=3pt] {$ past $} node[above=3pt] {$   $};
    \draw (5,0) node[below=3pt] {$ now $} node[above=3pt] {$ $};
    \draw [black, ultra thick ,decorate,decoration={brace,amplitude=5pt}, yshift=-4pt] (0,0.5)  -- (5,0.5)  node [black,midway,above=4pt,xshift=-2pt] {\footnotesize $\textbf{recently / in the last few days / since I arrived}$};
\end{tikzpicture} \\
In the same way we use the present perfect with $\textbf{today, this evening, this year}$ etc. when these periods are not finished at the time of speaking. \\
\begin{tikzpicture}[]
    % draw horizontal line
    \draw[] (0,0) -- (5,0);

    % draw nodes
    \draw (0,0) node[below=3pt] {$ past $} node[above=3pt] {$   $};
    \draw (5,0) node[below=3pt] {$ now $} node[above=3pt] {$ $};
    \draw [black, ultra thick ,decorate,decoration={brace,amplitude=5pt}, yshift=-4pt] (0,0.5)  -- (6,0.5)  node [black,midway,above=4pt,xshift=-2pt] {\footnotesize $\textbf{today}$};
\end{tikzpicture} \\
We say `It's the (first) time something $\textbf{has happened}$'. For example:
\begin{itemize}
    \item[$\square$] It's the first time he $\textbf{has driven}$ a car. (not drives)
    \item[$\square$] He $\textbf{hasn't driven}$ a car $\textbf{before}$.
    \item[$\square$] He $\textbf{has never driven}$ a car $\textbf{before}$.
\end{itemize}
In the same way we say:
\begin{itemize}
    \item[$\square$] Sarah has lost her passport again. This is the second time this $\textbf{has happened}$. ($\textit{not}$ happens)
    \item[$\square$] Andy is phoning his girlfriend again. It's the third time he$\textbf{'s phoned}$ her $\textbf{this evening}$.
\end{itemize}

\section{Present perfect continuous (I have been doing)}
\label{Present perfect continuous}
$\textbf{have/has been}$ + $\textbf{-ing}$ is the $\textit{present perfect continuous}$
\begin{flushleft}
\newcommand{\tabincell}[2]{\begin{tabular}{@{}#1@{}}#2\end{tabular}}
\begin{tabular}{ | l | l | l | l | }
 \hline
 \tabincell{c}{I/we/they/you\\ he/she/it} & \tabincell{l}{\textbf{have}\\ \textbf{has}} & \tabincell{l}{\textbf{been}} & \tabincell{l}{\textbf{doing}\\ \textbf{working}\\ \textbf{learning} etc.} \\
 \hline
\end{tabular}
\end{flushleft}
We use the present perfect continuous for an activity that has recently stopped or just stopped. \\
You can use the present perfect continuous for repeated actions.

\section{Present tenses (I am doing / I do) for the future}
\label{Present tenses (I am doing / I do) for the future}
$\textbf{I'm doing}$ something (tomorrow etc.) = I have already decided and arranged to do it. \\
We do not normally use $\textbf{will}$ to talk about what we have arranged to do. \\
We also use the present continuous for an action $\textit{just before you start to do it}$. This happens especially with verbs of movement($\textbf{go/come/leave}$ etc.). For example:
\begin{itemize}
    \item[$\square$] I'm tired. I$\textbf{'m going}$ tobed now. Goodnight. ($\textit{not}$ I go to bed now)
    \item[$\square$] `Tina, are you ready yet?' `Yes, I$\textbf{'m coming.}$' ($\textit{not}$ I come)
\end{itemize}
We use the present simple when we talk about timetables and programmes (for example, transport or cinema times). \\
You can use the present simple to talk about people if their plans are fixed like a timetable. \\
But the continuous is more usual for other personal arrangements. \\
Compare: \\
Present continuous
\begin{itemize}
    \item[$\square$] What time $\textbf{are}$ you $\textbf{arriving}$?
    \item[$\square$] I$\textbf{'m going}$ to the cinema this evening.
\end{itemize}
Present simple
\begin{itemize}
    \item[$\square$] What time $\textbf{does}$ the train $\textbf{arrive}$?
    \item[$\square$] The film $\textbf{starts}$ at 8.15.
\end{itemize}
When you talk about appointments, lessons, exams etc., you can use $\textbf{I have}$ or $\textbf{I've got}$
\begin{itemize}
    \item[$\square$] $\textbf{I have}$ an exam next week.\qquad or \qquad$\textbf{I've got}$ an exam next week.
\end{itemize}
